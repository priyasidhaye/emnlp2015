\section{Interaction with Formality}

As seen in the results of the analyses performed in \secref{sec:analysis}, the tweets have little in common with the articles they are related to. The analyses are based on the ROUGE-1,2 and L like calculations. This shows that extractive summarization algorithms cannot be directly applied to articles to generate tweets. 

To tie in the results of the findings above with some intuitive notions about the text and see how formality interacts with the results, we also calculated the formality of the articles. This formality score was correlated with the longest common subsequence. To achieve this, the degree of formality of the text was calculated with the help of \newcite{brooke2013multi}. The formality lexicon was generated by analyzing the stylistics of text and can be used to measure formality of a given text. They calculate formality scores for words and sentences by training a model on a large corpus based on the appearance of words in specific documents. Their model represents words as vectors and the formal and informal seeds appear in opposite halves of the graphs, suggesting that we can use these seeds to determine if an article is formal or informal. The lexicon consists of words and phrases and the degree of formality for their occurrence. Thus, more formal words are marked on a positive scale and informal words like those occurring in colloquial language are marked on a negative scale. The degree of formality was calculated using this lexicon. Let the set of formality expressions from the lexicon be $L$, where $L$ contains the formality expression, $e$, and the formality score for this expression is $L(e)$. Let the set of all substrings from the article $\textit{substrings}(a)$ be $S$. Then, the formality score $f$ for a tweet $t$ and article $a$ is the number of formality expressions per 10 words in article, described as  

\begin{equation}
f = \frac{\sum\limits_{e \in L, e \in S} L(e)}{| unigrams(a) |} * 10
\end{equation}

The formality lexicon gave positive weights for formal expressions and negative for informal expressions. After calculating the formality weights for all articles, it was observed that they all had a total negative normalized weight, meaning a lot more informal expressions were getting matched. Hence, we used just the formal word occurrences for calculating the weight. Thus, above a certain cut-off weight, the article could be considered formal, else would be considered informal. To make sure these formality scores intuitively made sense, we calculated the average formality score for each hashtag used in the search during data extraction and ordered them, shown in \tabref{tab:formal}


\begin{table}[htbp]
\centering
\begin{tabular}{|l|l|}
\hline
Lowest  & Highest  \\ \hline
\#theforceawakens       & \#KevinVickers           \\
\#TaylorSwift           & \#erdogan                \\
\#winteriscoming        & \#apec                  \\ \hline
\end{tabular}
\captionof{table}{Table of hashtags(broadly, topics) with highest and lowest formality according to the lexicon.}
\label{tab:formal}
\end{table}

This formality score for each article was then correlated with the percentage of match obtained using the longest common subsequence algorithm. The Pearson correlation value was 0.41, with a p-value of 7.08e-66. The p-value justifies that we can reject the null hypothesis, and say with confidence that there is a correlation between the formality scores and the ROUGE-L scores of the tweets and articles. Hence, we can say that the more formal the subject or the article, there are higher chances of the tweet being extracted directly from the article. \tabref{tab:formal} gives an example of the formality of the article, which is a low 4.2 formality words per 10 words, where the tweet is not extracted from the article, but rephrased from the article instead.

\begin{table}[htbp]
\centering
\begin{tabular}{|p{0.1\linewidth}|p{0.8\linewidth}|}
\hline
Tweet &  @globetoronto: Why Buffalo got clobbered with snow and Toronto did not. \#weather \#snowstorm http://t.co/gcwwoDPZmX... http://t.co/BXY7EH6F3u" \\ \hline
Title & What caused Buffalo’s massive snow and why Toronto got lucky \\  \hline
Text  & Torontonians have long been the butt of jokes about calling in the army every time a few snow flurries whip by... \\ \hline
\end{tabular}
\captionof{table}{Example of a tweet, title of the article where the formality of the article is over the mean, and the tweet is extracted from the article.}
\label{tab:formal}
\end{table}

\section{Discussions}

A way to get around extractive summarizations for tweet generation and move towards more abstractive solutions is to learn to be able to classigy intent, or the purpose of sharing the tweets. Studies on classifying user intents in tweets are interpreted in different ways. \newcite{banerjee2012towards} analyze real time data to detect presence of intents in tweets. \newcite{wang2015mining} classify intents as food and drink, travel, career and so on, ones that can directly be used as intents for purchasing and can be utilized for advertisements. They also focus on finding tweets with intent and then classifying those. \newcite{gomez2014content} use features from text and stylistics to determine user intentions, which are classified as news report, opinion, publicity and so on. \newcite{mohammad2013identifying} study the classification of user intents specifically for tweets related to elections. They study one election and classify tweets as ones that agree or disagree with the candidate, ones that are meant for humor, support and so on. 

However, these definitions of intent will still not be sufficient for tweet generation. For this purpose, intent would be the reason the user chose to share the article with that particular text. This would include reasons like support some cause, promote a product or an article, agree or disagree with an event, or express an opinion about it. Identifying these intents will help provide parameters for generating tweets, which can then be used towards abstractive summarization. 