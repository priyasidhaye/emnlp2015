\section{Introduction}
\label{sec:intro}
With the rise in popularity of social media, message broadcasting sites such as Twitter and other microblogging services have become an important means of communication, with an estimated 500 million tweets being written every day\footnote{https://about.twitter.com/company}. In addition to individual users, various organizations and public figures such as newspapers, government officials and entertainers have established themselves on social media in order to disseminate information or promote their products. 

While there has been recent progress in the development of Twitter-specific POS taggers, parsers, and other tweet understanding tools \cite{owoputi-etal-2013,kong-etal-2014}, there has been little work on methods for generating tweets, despite the utility this would have for users and organizations. 

In this paper, we study the generation of the particular class of tweets that contain a link to an external web page that is composed primarily of text. This class of tweets, which we call \emph{indicative tweets}, represents a large subset of tweets overall, constituting more than half of the tweets in our data set.  Indicative tweets would appear to be the easiest to handle using current methods in text summarization, because there is a clear source of input from which a tweet could be generated. In effect, the tweet would be acting as an indicative summary of the article being linked to, and it would seem that existing methods in summarization can be applied. 

There has in fact been some work along these lines, within the framework of extractive summarization. \newcite{lofi2012iparticipate} describe a system to generate tweets from local government records through keyphrase extraction. \newcite{lloret2013towards} compares various extractive summarization algorithms applied on Twitter data to generate tweets from documents. 

Lofi and Krestel do not provide a formal evaluation of their model, while Lloret and Palomar compared overlap between system-generated and user-generated tweets using ROUGE 
\cite{lin2004rouge}. Unfortunately, they also show that there is little correlation between ROUGE scores and the perceived quality of the tweets when rated by human users for indicativeness and interest. More scrutiny is required to determine whether the wholesale adoption of methods and evaluation schemes from extractive summarization is justified.

Beyond issues of evaluation measures, it is also unclear whether extraction is the strategy employed by human tweeters. One of the original motivations behind extractive summarization was the observation that human summary writers tended to extract snippets of key phrases from the source text \cite{mani-2001}.  And while it may be true that an automatic tweet generation system need not necessarily follow the same approach to writing as human tweeters, it is still necessary to know what proportion of tweets could be accounted for in an extractive summarization paradigm.

With indicative tweets, an additional issue arises in that the genre of the source text is not constrained. Thus, it may be a news article or an informal blog post. This may be vastly different from the desired formality of tweet itself, and thus, a genre-appropriate extract may not be available.

We begin to address the above issues through a study that examines to what extent tweet generation can be viewed as an extractive summarization problem. We extracted a data set of indicative tweets containing a link to an external article, including the documents linked to by the tweets. We used this data and applied unigram, bigram and LCS (longest common subsequence) matching techniques inspired by ROUGE to determine what proportion of tweets can be found in the linked article. Even with the permissive unigram match measure, we find that well under half of the tweet can be found in the linked article. We also use stylistic analysis on the articles to examine the role that genre differences between the source text and the target tweet play, and find that it is easier to extract tweets from more formal articles than less formal ones.

Our results point to the need for the development of a methodology for indicative tweet generation, rather than to expropriate the extractive summarization paradigm that was developed mostly on news text. Such a methodology will ideally be sensitive to stylistic factors as well as the underlying intent of the tweet.