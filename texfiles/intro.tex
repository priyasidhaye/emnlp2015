\section{Introduction}
\improvement{First two paragraphs - order and glue}
Summarization techniques developed till now can be broadly classified as extractive or abstractive \cite{hahn2000challenges}. Extractive summarization identifies keywords and phrases from the original text and strings them together to form a summary, while abstractive summarization describes the content of the document in more general terms.

With the rise in popularity of social media, message broadcasting sites have become the new means of communication, voicing opinions, broadcasting news, promotions and so on. Newspapers, channels, movies, government officials, entertainers, have all established themselves on social media and use it regularly for broadcasting news and promoting their products. Twitter is such a public message broadcasting service. Since all posts on the website are a limited length, 140 characters, it has been termed as microblogging. The messages, called tweets, convey precise information about a given topic in a limited length. The nature of this website has made it popular for global discussions on current goings-on in the world, with an estimated 200 million tweets being tweeted per day.

The tweets are often used as a link to a new web page, that has more detailed information about the topic being discussed. This set up suggests that the web page, which might contain videos, images, or articles, blogs, and so on is being promoted with the use of the tweet. In the case of articles, intuitively, the tweet seems to be an indicative, informative, or critical summary of the article being promoted. Hence, it might seem that the problem of tweet generation in this context can be looked at as an extractive summarization problem where the linked article is the source text and the tweet is the generated summary. 

\newcite{lloret2013towards} and \newcite{lofi2012iparticipate} both study this same problem of generating tweets using summarization methods. While the former compares various extractive summarization algorithms with Twitter data to generate tweets from documents, the latter suggests a system to generate a tweet using documents from local government records. 

To validate modelling of tweet generation as an extractive summarization problem, we extracted such a data set from Twitter, also extracting linked documents through the tweets. We used this data and applied unigram, bigram and LCS (longest common subsequence) matching techniques to show that to generate tweets from related articles we need a more involved approach than blindly using extractive summarization algorithms. We also use stylistic analysis on the articles to explain the results obtained.